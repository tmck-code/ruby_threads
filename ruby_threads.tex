\documentclass{article}
\usepackage{latexsym}
\usepackage[utf8]{inputenc}
\usepackage{tikz} \usetikzlibrary{matrix,shapes,arrows,positioning,chains}
\title{Readme: Lexer \& FullContact Enrichment}
\author{tom.mckeesick@lexer.io}
\date{2016-03-18}

\tikzstyle{decision} = [diamond, draw, fill=white!20, text width=4.5em, text badly centered, node distance=3cm, inner sep=0pt]
\tikzstyle{block} = [rectangle, draw, fill=white!20, text width=10em, text centered, rounded corners, minimum height=4em, node distance=2cm]
\tikzstyle{line} = [draw, -latex']
\tikzstyle{cloud} = [draw, ellipse,fill=green!10, text centered, node distance=5cm, minimum width=4em, minimum height=4em]

\begin{document}
\maketitle

\tableofcontents
\cleardoublepage

\part{Introduction}
\section{Overview}
\subsection{Intended Function}
The fullcontact repository is the collection of all programs and scripts related to the unification of Lexer identities with the FullContact API.
Currently, it has just a few responsibilities:
\begin{enumerate}
  \item Consume identity objects from a queue.
  \item Format API request depending on the available attributes of each identity.
  \item Send a batch query to the FullContact API.
 \item Extract the new attributes for each identity from the API response, and create those attributes for each identity.
 \item Provide a report detailing the number of FullContact attributes that have been used
\end{enumerate}

\cleardoublepage

\part{Structure}
\section{Structure Overview}
\subsection{Main functionality contained in /lib}
\begin{description}
  \item[queue-consumer] A thread-safe background worker that consumes identities in batches from a Rabbit MD queue
  \item[query] Receives identities in batches from the \textit{queue-consumer}, extracts the attributes need for the FullContact API query, sends the queries and passes along the response.
  \item[process-responses] Receives the JSON API response (and the original identity batch) from \textit{query-api}. 
It then extracts the required attributes and processes them accordingly with the namespace com.fullcontact. It then creates the new identities amd passes them to enrich.
 \item[enrich] Receives the batch of updated identities from \textit{process-responses} and inserts them into the database.
 \item[report] Takes a recurring snapshot of all identities and produces a report om the number of com.fullcontact namespace attributes.
\end{description}

\section{Class Structure}

\begin{tikzpicture}[node distance = 2cm, auto]
 \node [cloud, fill=red!30] (queue) 
       {\textbf{identity-queue}};
 \node [block, below of=queue] (worker) 
       {queue-consumer};
 \node [cloud, right of=worker] (worker-helper)
       {\textit{parse-identity}};
 \node [block, below of=worker] (query) 
       {query-fullcontact};
 \node [cloud, right of=query] (query-helper)
       {\textit{query-helper}};
 \node [block, below of=query] (process) 
       {process-responses};
 \node [cloud, left of=process] (process-helper)
       {\textit{response-details}};
 \node [cloud, right of=process] (attr-helper)
       {\textit{attribute-helper}};
 \node [block, below of=process] (enrich)
       {enrich};
 \node [cloud, right of=enrich] (id-helper)
       {\textit{parse-identity}};
 \node [cloud, left of=enrich] (enrich-helper)
       {\textit{enrichment-helper}};
 \node [cloud, below of=enrich, fill=blue!10] 
       (id-api) {identity API};
       (stop) {stop};
% Draw edges
 \path [line] (queue) -- (worker);
 \path [line] (worker) -- (worker-helper);
 \path [line] (worker-helper) -- (worker);
 \path [line] (worker) -- (query);
 \path [line] (query) -- (query-helper);
 \path [line] (query-helper) -- (query);
 \path [line] (query) -- (process);
 \path [line] (process) -- (process-helper);
 \path [line] (process-helper) -- (process);
 \path [line] (query) -- (process);
 \path [line] (process) -- (attr-helper);
 \path [line] (attr-helper) -- (process);
 \path [line] (process) -- (enrich);
 \path [line] (enrich) -- (enrich-helper);
 \path [line] (enrich-helper) -- (enrich);
 \path [line] (enrich) -- (id-helper);
 \path [line] (id-helper) -- (enrich);
 \path [line] (enrich) -- (id-api);

\end{tikzpicture}

\cleardoublepage
\part{Disscussion}

\end{document}
